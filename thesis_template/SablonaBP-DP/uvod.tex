\chapter*{Úvod}
\addcontentsline{toc}{chapter}{Úvod}

Úvod je povinnou částí bakalářské/diplomové práce. Úvod je uvedením do tématu. Zvolené téma rozvádí, stručně ho zasazuje do souvislostí (může zde být i popis motivace k sepsání práce) a odpovídá na otázku, proč bylo téma zvoleno. Zasazuje téma do souvislostí a zdůvodňuje jeho nutnost a aktuálnost řešení. Obsahuje explicitně uvedený cíl práce. Text cíle práce je shodný s textem, který je uveden v zadání bakalářské práce, tj. s textem, který je uveden v systému InSIS a který je také uveden v části Abstrakt.

Součástí úvodu je také stručné představení postupu zpracování práce (detailně je metodě zpracování věnována samostatná část vlastního textu práce). Úvod může zahrnovat i popis motivace k sepsání práce.

Úvod k diplomové práci musí být propracovanější -- podrobněji to je uvedeno v Náležitostech diplomové práce v rámci Intranetu pro studenty FIS.

Následuje několik ukázkových kapitol, které doporučují, jak by se měla bakalářská/diplomová práce sázet. Primárně popisují použití \TeX{}ové šablony, ale obecné rady poslouží dobře i~uživatelům jiných systémů.
