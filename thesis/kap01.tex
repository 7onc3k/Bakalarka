\chapter{Vymezení problému a cílů práce}
\label{kap:vymezeni}

\section{Motivace}

\begin{draft}
Studie společnosti METR.ORG ukazuje, že LLM zkušené vývojáře spíše zpomaluje. S rychlým vývojem schopností modelů se situace pravděpodobně mění. Ale to neznamená, že je LLM samo o sobě dostatečné k vypracování dlouhotrvajících úkolů. Není to problém pouze LLM, když projekt roste, bývá těžší jej rozšiřovat jak pro člověka, tak i pro LLM. AI programování tenhle problém ještě více prohlubuje. Vývojáři přicházejí o kontext a hlubokou znalost kódové základny (codebase), zatímco velké jazykové modely (LLM) jsou limitovány omezenou pamětí (context window). Jak nastavit harness/scaffolding tak, aby v tom mohli fungovat agenti a lidé to stále měli pod kontrolou?
\end{draft}

\section{Cíle práce}

\begin{draft}
\begin{enumerate}
    \item Popsat jak se řízení SWE projektů mění v kontextu agentních systémů (teoretický rámec)
    \item Navrhnout a implementovat experimentální prostředí (case study: systém upomínek faktur)
    \item Prozkoumat vliv různých nastavení scaffoldingu na schopnost agenta provést kvalitní práci
    \item Identifikovat jaký kontext je pro agenty klíčový a jak instrukce/procesy ovlivňují schopnost agenta tento kontext vytvářet a využívat
\end{enumerate}
\end{draft}

\section{Rozsah práce}

\begin{raw}
\textbf{Poznámky k propojení:}
\begin{itemize}
    \item Řízení vyžaduje holistický pohled (vidět celek, ne jen část) - proto celý SDLC, ne jedna fáze
    \item Billing Reminder Engine jako case study: malý projekt, ale reálné nuance (state machine, business days, edge cases)
    \item Deterministická logika (stejný vstup = stejný výstup) + jasná pravidla správnosti = objektivně testovatelné
    \item Hraniční případy (víkendy, svátky, grace periods) = místa kde lze testovat kvalitu agentní práce
\end{itemize}

\textbf{TODO: Vysvětlit termíny:}
\begin{itemize}
    \item state machine - ?
    \item edge cases / hraniční případy - ?
    \item SDLC - ?
\end{itemize}

\textbf{Scope SDLC (k diskuzi):}
\begin{itemize}
    \item Primární plán: celý SDLC včetně deployment/maintenance
    \item Fallback: zúžit na implementation + testing (hlavní doména coding agents)
    \item Poznámka: i impl + testing má feedback loop (implementace → testy → chyba → úprava) = stále systémový pohled
\end{itemize}
\end{raw}

\begin{draft}
Práce pokrývá celý SDLC na jednoduchém projektu. Součástí je vlastní implementace, která poskytuje hloubku porozumění potřebnou pro návrh experimentů.

\subsection*{Práce se zaměřuje na:}
\begin{itemize}
    \item Nastavení a použití existujících nástrojů (GitHub, CLI agents)
    \item Exploratory case study - hledání vzorů a doporučení
\end{itemize}

\subsection*{Práce se nezaměřuje na:}
\begin{itemize}
    \item Programování nových nástrojů od nuly
    \item Porovnávání různých LLM modelů
    \item Porovnávání různých programovacích jazyků
    \item Vývoj produkčního nástroje/produktu
\end{itemize}
\end{draft}
