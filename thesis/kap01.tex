\chapter{Vymezení problému a cílů práce}
\label{kap:vymezeni}

\section{Motivace}

\begin{draft}
Velké jazykové modely (LLM) dnes umí generovat kód, ale to neznamená, že umí vyvíjet software. Randomizovaná studie METR \cite{metr2025} na 16 zkušených vývojářích a~246 úlohách ukázala, že s~AI nástroji byli vývojáři o~19\,\% pomalejší --- přestože sami odhadovali zrychlení o~24\,\%. Když projekt roste, bývá těžší jej rozšiřovat --- jak pro člověka, tak pro LLM. Vývojáři přicházejí o~kontext a hlubokou znalost kódové základny, zatímco agenti jsou limitováni omezeným context window. Otázka tedy není, co modely umí, ale jak je zasadit do vývojového procesu tak, aby produkovaly kvalitní výstup.

Ukazuje se, že odpověď leží spíš v~instrukcích než v~samotných modelech. Breunig \cite{breunig2025} zjistil, že změna promptu mění chování agenta výrazněji než výměna modelu. Lulla et al. \cite{lulla2026} dokládají, že přidání architektonických konvencí do instrukčního souboru zkrátí dobu běhu o~28\,\% a sníží spotřebu tokenů o~20\,\%. Scaffolding --- struktura instrukcí a procesních pravidel, která agenta provází vývojem --- má na výsledek zásadní vliv. Ale nikdo systematicky nezkoumal, \emph{které} složky tohoto scaffoldingu jsou pro dodržování softwarově-inženýrských praktik nezbytné a které jsou zbytečné.

Tato práce zkoumá právě to: jak musí být strukturovány instrukce pro AI coding agenta, aby dodržoval specifikací řízený vývoj, test-driven development a granulární správu verzí --- a co z~těch instrukcí je skutečně kritické.
\end{draft}

\begin{raw}
Studie společnosti METR.ORG ukazuje, že LLM zkušené vývojáře spíše zpomaluje. S rychlým vývojem schopností modelů se situace pravděpodobně mění. Ale to neznamená, že je LLM samo o sobě dostatečné k vypracování dlouhotrvajících úkolů. Není to problém pouze LLM, když projekt roste, bývá těžší jej rozšiřovat jak pro člověka, tak i pro LLM. AI programování tenhle problém ještě více prohlubuje. Vývojáři přicházejí o kontext a hlubokou znalost kódové základny (codebase), zatímco velké jazykové modely (LLM) jsou limitovány omezenou pamětí (context window). Jak nastavit harness/scaffolding tak, aby v tom mohli fungovat agenti a lidé to stále měli pod kontrolou?
\end{raw}

\section{Cíle práce}

\begin{draft}
\begin{enumerate}
    \item Popsat, jak se řízení softwarových projektů mění v~kontextu AI agentů, a zmapovat co víme o~vlivu instrukcí na jejich chování.
    \item Iterativně navrhnout instrukce, které dovedou AI coding agenta k~dodržování vývojového workflow na case study projektu (systém upomínek faktur).
    \item Zjistit, které složky těchto instrukcí jsou pro dodržování workflow nezbytné, které jsou zbytečné, a jaký kontext agent potřebuje k~tomu, aby vytvářel a využíval projektové artefakty.
\end{enumerate}
\end{draft}

\begin{raw}
\begin{enumerate}
    \item Popsat jak se řízení SWE projektů mění v kontextu agentních systémů (teoretický rámec)
    \item Navrhnout a implementovat experimentální prostředí (case study: systém upomínek faktur)
    \item Prozkoumat vliv různých nastavení scaffoldingu na schopnost agenta provést kvalitní práci
    \item Identifikovat jaký kontext je pro agenty klíčový a jak instrukce/procesy ovlivňují schopnost agenta tento kontext vytvářet a využívat
\end{enumerate}
\end{raw}

\section{Rozsah práce}

\begin{draft}
Práce zkoumá vliv struktury instrukcí na chování jednoho AI coding agenta při vývoji z~formální specifikace. Case study je systém upomínek faktur --- projekt s~deterministickou logikou, jasnými pravidly správnosti a objektivně ověřitelnými výstupy. Experiment pokrývá fáze requirements, design a implementation.

\subsection*{Práce se zaměřuje na:}
\begin{itemize}
    \item Vliv struktury a obsahu instrukcí na to, jak agent dodržuje vývojový workflow
    \item Iterativní návrh instrukcí metodou Design Science Research
    \item Identifikaci toho, co z~instrukcí je potřeba a co ne, prostřednictvím ablace
\end{itemize}

\subsection*{Práce se nezaměřuje na:}
\begin{itemize}
    \item Porovnávání LLM modelů
    \item Porovnávání programovacích jazyků a frameworků
    \item Projekty s~převážně subjektivními výstupy (UI, kreativní tvorba, ML)
\end{itemize}
\end{draft}

\begin{raw}
\textbf{Poznámky k propojení:}
\begin{itemize}
    \item Řízení vyžaduje holistický pohled (vidět celek, ne jen část) - proto celý SDLC, ne jedna fáze
    \item Billing Reminder Engine jako case study: malý projekt, ale reálné nuance (state machine, business days, edge cases)
    \item Deterministická logika (stejný vstup = stejný výstup) + jasná pravidla správnosti = objektivně testovatelné
    \item Hraniční případy (víkendy, svátky, grace periods) = místa kde lze testovat kvalitu agentní práce
\end{itemize}

\textbf{TODO: Vysvětlit termíny:}
\begin{itemize}
    \item state machine - ?
    \item edge cases / hraniční případy - ?
    \item SDLC - ?
\end{itemize}

\textbf{Scope SDLC (k diskuzi):}
\begin{itemize}
    \item Primární plán: celý SDLC včetně deployment/maintenance
    \item Fallback: zúžit na implementation + testing (hlavní doména coding agents)
    \item Poznámka: i impl + testing má feedback loop (implementace → testy → chyba → úprava) = stále systémový pohled
\end{itemize}
\end{raw}
