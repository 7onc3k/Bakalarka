%% Toto platí v případě použití samostatné bibliografické databáze
\printbibliography[title={\bibname},heading={bibintoc}]

%% ============================================================================
%% RAW ZDROJE K ZPRACOVÁNÍ DO BIBTEX
%% ============================================================================

\begin{raw}
\textbf{Zdroje k zpracování do BibTeX} (původně z notes/sources.md)

\textbf{1. PRIMÁRNÍ ZDROJE (Frameworky a Metodiky)}

GITHUB NEXT. Spec Kit: Spec-Driven Development for AI Agents [online]. 2024.
\url{https://github.com/github/spec-kit}
-- Klíčový zdroj pro koncept "Spec-Driven Development".

BMAD-CODE-ORG. BMAD METHOD: Breakthrough Method for Agile AI-Driven Development [online]. GitHub, 2025.
\url{https://github.com/bmad-code-org/BMAD-METHOD}
-- Metodika pro řízení AI agentů v agilním vývoji.

ANTHROPIC. Effective Harnesses for Long-Running Agents [online]. Anthropic Engineering Blog, 2024.
\url{https://www.anthropic.com/engineering/effective-harnesses-for-long-running-agents}
-- Technický popis problémů s dlouhodobou pamětí agentů.

METR. Model Evaluation and Threat Research [online]. 2024.
\url{https://metr.org/}
-- Standardy pro hodnocení bezpečnosti a schopností modelů.

THEDOTMACK. claude-mem: Persistent Memory System for Claude Code [online]. GitHub, 2025.
\url{https://github.com/thedotmack/claude-mem}
-- Příklad implementace hooks v CLI agentech - persistentní paměť přes session.

\textbf{1.2 SYSTÉMOVÉ MYŠLENÍ V SWE}

PETKOV, Doncho et al. Information Systems, Software Engineering, and Systems Thinking: Challenges and Opportunities.
International Journal of Information Technologies and Systems Approach.
\url{https://www.igi-global.com/gateway/article/2534}
-- Mapuje historii systémového přístupu v IS a SWE. Propojení systémového myšlení s praxí SWE je stále nedotažené.

MONAT, Jamie a GANNON, Thomas. Systems Thinking: A Review and Bibliometric Analysis.
MDPI Systems, 2020.
\url{https://www.mdpi.com/2079-8954/8/3/23}
-- Přehled co systémové myšlení je a kde se používá. Interdisciplinární - SWE, management, vzdělávání.

ALHARTHI, Sultan et al. A Systems Thinking Approach to Improve Sustainability in Software Engineering.
MDPI Sustainability, 2023.
\url{https://www.mdpi.com/2071-1050/15/11/8766}
-- Praktická aplikace - dívají se na vývoj jako systém (developeři, zákazníci, stakeholders).

\textbf{2. ODBORNÉ STUDIE}

\textbf{2.1 Přehledové studie (Surveys) - LLM agenti v SE}

LIU, Junwei et al. Large Language Model-Based Agents for Software Engineering: A Survey.
arXiv preprint arXiv:2409.02977. 2024.
\url{https://arxiv.org/abs/2409.02977}
-- Komplexní přehled 106 prací o LLM agentech v SE, kategorizace z pohledu SE i agentů.

JIN, Haolin et al. From LLMs to LLM-based Agents for Software Engineering: A Survey of Current, Challenges and Future.
arXiv preprint arXiv:2408.02479. 2024.
\url{https://arxiv.org/abs/2408.02479}
-- Pokrývá requirements, code generation, testing, maintenance - celý SDLC.

Comprehensive Survey on Benchmarks and Solutions in Software Engineering of LLM-Empowered Agentic System.
arXiv preprint arXiv:2510.09721. 2025.
\url{https://arxiv.org/html/2510.09721}
-- Přes 150 paperů, taxonomie řešení a benchmarků.

A Survey on Code Generation with LLM-based Agents.
arXiv preprint arXiv:2508.00083. 2025.
\url{https://arxiv.org/abs/2508.00083}
-- Single-agent a multi-agent architektury, aplikace napříč SDLC.

\textbf{2.2 Agentic Software Engineering - Klíčové práce}

Agentic Software Engineering: Foundational Pillars and a Research Roadmap.
arXiv preprint arXiv:2509.06216. 2025.
\url{https://arxiv.org/html/2509.06216v2}
-- Přehodnocení SE pro spolupráci člověk-agent. Framework podobný SAE úrovním autonomie.
Rozlišuje SE 2.0 (AI-augmented) vs SE 3.0 (Agentic SE).

AKBAR, Muhammad Azeem et al. Agentic AI in Software Engineering: Practitioner Perspectives Across the Software Development Life Cycle.
SSRN. 2025.
\url{https://papers.ssrn.com/sol3/papers.cfm?abstract_id=5520159}
-- Rozhovory s 21 experty, pokrývá celý SDLC. Zjištění: agenti redefinují hranice mezi fázemi SDLC.

Autonomous Agents in Software Development: A Vision Paper.
Springer, 2024.
\url{https://link.springer.com/chapter/10.1007/978-3-031-72781-8_2}
-- 12 LLM agentů spolupracujících na celém SDLC.

\textbf{2.3 Evaluace a produktivita}

METR. Measuring the Impact of Early-2025 AI on Experienced Open-Source Developer Productivity. 2025.
\url{https://metr.org/blog/2025-07-10-early-2025-ai-experienced-os-dev-study/}
-- RCT studie: zkušení vývojáři s AI jsou o 19\% pomalejší - překvapivé zjištění.

\textbf{2.4 Metriky kvality software a AI agentů}

ISO/IEC. ISO/IEC 25010:2023 - Systems and software Quality Requirements and Evaluation (SQuaRE) — Product quality model. 2023.
\url{https://www.iso.org/standard/78176.html}
-- Industry standard pro kvalitu software. 8 charakteristik, Functional Suitability obsahuje Completeness, Correctness, Appropriateness.

ISO 25000. ISO 25010 Software Quality Model. 2023.
\url{https://iso25000.com/index.php/en/iso-25000-standards/iso-25010}
-- Přehledný popis ISO 25010 modelu kvality.

LXT. AI Agent Evaluation: Comprehensive Framework for Measuring Agent Performance. 2024.
\url{https://www.lxt.ai/blog/ai-agent-evaluation/}
-- Moderní framework pro evaluaci AI agentů: Task completion, Accuracy, Safety/trust (policy compliance, transparency), Tool usage.

Weights \& Biases. AI Agent Evaluation: Metrics, Strategies, and Best Practices. 2024.
\url{https://wandb.ai/onlineinference/genai-research/reports/AI-agent-evaluation-Metrics-strategies-and-best-practices--VmlldzoxMjM0NjQzMQ}
-- Praktický průvodce metrikami pro AI agenty.

\textbf{2.5 Základní LLM studie}

JIMENEZ, Carlos E. et al. SWE-bench: Can Language Models Resolve Real-world Github Issues?
In: The Twelfth International Conference on Learning Representations (ICLR). 2024.
\url{https://arxiv.org/abs/2310.06770}
-- Hlavní benchmark pro hodnocení schopností programovacích agentů.

LIU, Nelson F. et al. Lost in the Middle: How Language Models Use Long Contexts.
arXiv preprint arXiv:2307.03172. 2023.
\url{https://arxiv.org/abs/2307.03172}
-- Klíčová studie vysvětlující, proč pouhé zvětšení kontextového okna nestačí.

VASWANI, Ashish et al. Attention Is All You Need.
Advances in Neural Information Processing Systems, 2017.
\url{https://arxiv.org/abs/1706.03762}
-- Základní paper definující Transformer architekturu a mechanismus pozornosti (self-attention).

WEI, Jason et al. Chain-of-Thought Prompting Elicits Reasoning in Large Language Models.
Advances in Neural Information Processing Systems, 2022.
\url{https://arxiv.org/abs/2201.11903}
-- Základ pro techniky prompt engineeringu používané v práci.

\textbf{3. TEORIE SOFTWAROVÉHO INŽENÝRSTVÍ A ARCHITEKTURY}

RICHARDS, Mark a FORD, Neal. Fundamentals of Software Architecture: An Engineering Approach.
O'Reilly Media, 2020. ISBN 978-1492043454.
-- Moderní přehled architektonických stylů a charakteristik ("ilities").

FOWLER, Martin. Patterns of Enterprise Application Architecture.
Addison-Wesley Professional, 2002. ISBN 978-0321127426.
-- Katalog základních návrhových vzorů pro podnikové aplikace.

KHONONOV, Vlad. Learning Domain-Driven Design: Aligning Software Architecture and Business Strategy.
1. vyd. O'Reilly Media, 2021. ISBN 978-1098100131.
-- Definuje pojmy jako "Bounded Context" a "Ubiquitous Language", které jsou analogií pro kontext LLM.

ISO/IEC. ISO/IEC/IEEE 42010:2011 Systems and software engineering — Architecture description. 2011.
-- Mezinárodní standard definující základní pojmy popisu architektury.

IEEE COMPUTER SOCIETY. SWEBOK: Guide to the Software Engineering Body of Knowledge.
Version 3.0. IEEE, 2014.
\url{https://www.swebok.org/}
-- Standardní taxonomie softwarového inženýrství.

BROWN, Simon. The C4 model for visualising software architecture. 2024.
\url{https://c4model.com/}
-- Metodika pro hierarchický popis architektury, vhodná pro strojové zpracování.

ARC42. arc42 Template for Software Architecture Documentation. 2024.
\url{https://arc42.org/}
-- Pragmatická šablona pro strukturování architektonické dokumentace.

\textbf{4. DALŠÍ ZDROJE (Historický kontext a Procesy)}

NATO SCIENCE COMMITTEE. Software Engineering: Report on a conference sponsored by the NATO Science Committee.
Garmisch, Germany, 1968.
-- Historický kontext vzniku disciplíny.

KAUR, Rupinder a SENGUPTA, Jyotsna. Software Process Models and Analysis on Failure of Software Development Projects.
In: arXiv preprint arXiv:1306.1068. 2013.
\end{raw}
