\chapter{Vyhodnocení a diskuse}
\label{kap:vyhodnoceni}

\begin{raw}
[RAW] Kostra kapitoly --- zde bude interpretace výsledků z kap04.
\end{raw}

\section{Interpretace výsledků}

\begin{raw}
[RAW] Co výsledky znamenají --- propojení s výzkumnými otázkami.

\begin{itemize}
    \item Které složky scaffoldingu jsou nezbytné a které redundantní?
    \item Jaký kontext agent potřebuje k dodržování SWE praktik?
    \item Odpovědi na cíle práce (viz kap01)
\end{itemize}
\end{raw}

\section{Porovnání s literaturou}

\begin{raw}
[RAW] Zasazení výsledků do kontextu existujícího výzkumu.

\begin{itemize}
    \item Porovnání s Lulla et al. (AGENTS.md efekt)
    \item Porovnání s Breunig (prompt vs. model)
    \item Porovnání s ablačními studiemi (SWE-agent, CCA, RePrompt)
    \item Kde naše výsledky potvrzují / rozporují existující evidence
\end{itemize}
\end{raw}

\section{Limity a hrozby validity}

\begin{raw}
[RAW] Přiznání omezení studie.

\begin{itemize}
    \item Prompt sensitivity \cite{razavi2025} --- malá změna formulace může změnit výsledky
    \item Jeden model, jeden projekt --- analytická generalizace, ne statistická
    \item Single run per podmínka --- omezená reprodukovatelnost
    \item Case study generalizuje na teorii, ne na populaci \cite{yin2018}
\end{itemize}
\end{raw}

\section{Doporučení pro praxi}

\begin{raw}
[RAW] Praktické závěry pro vývojáře pracující s AI coding agenty.

\begin{itemize}
    \item Co zahrnout do AGENTS.md / instrukčních souborů
    \item Co je zbytečné / kontraproduktivní
    \item Minimální efektivní scaffolding
\end{itemize}
\end{raw}

\section{Náměty pro další výzkum}

\begin{raw}
[RAW] Future work.

\begin{itemize}
    \item Více modelů, více projektů
    \item Systematické porovnání formátů specifikace
    \item Multi-agent vs. single-agent scaffolding
    \item Longitudinální studie (více issues, evoluce projektu)
\end{itemize}
\end{raw}
