\chapter{Praktická část}
\label{kap:prakticka-cast}

\begin{raw}
[RAW] Tato kapitola popisuje provedení experimentu a jeho výsledky.
\end{raw}

\section{Pilotní iterace}
\label{sec:pilotni-vysledky}

\subsection{Konstrukce baseline instrukcí}
\label{sec:konstrukce-instrukci}

\begin{raw}
[RAW] Konstrukce výchozí verze AGENTS.md na základě empirických frameworků.

Výchozí instrukce staví na empirických zjištěních z~literatury:

\begin{itemize}
    \item \textbf{Strukturální principy} --- Mao et al. \cite{mao2025} identifikovali
          kategorii ``requirement'' promptů (role, cíle, omezení) jako nejefektivnější
          pro řízení agentního chování. Instrukce explicitně definují roli agenta,
          procesní požadavky a~quality gates.
    \item \textbf{TDD jako procesní scaffolding} --- spec-first TDD \cite{mathews2024}
          zajišťuje, že testy vychází ze specifikace (ne z~pozorování kódu).
          Instrukce vyžadují sekvenci test $\to$ implementace $\to$ commit.
    \item \textbf{Dekompozice} --- rozložení práce do sub-issues před kódem
          snižuje kognitivní zátěž a~umožňuje branch-per-issue workflow.
    \item \textbf{Anti-patterns z~pilotu} --- instrukce explicitně zakazují
          chování pozorované v~předchozích iteracích (sycophantic test
          modification, monolitické commity, přeskočení specifikace).
\end{itemize}

Konkrétní mapování: každý řádek AGENTS.md má odůvodnění
buď v~literatuře (citace), nebo v~diagnostice z~pilotního běhu (odkaz
na iteraci kde se problém projevil).
\end{raw}

\subsection{Baseline AGENTS.md}

\begin{raw}
[RAW] Kompletní znění výchozích instrukcí. Mapování exit kritérií na konkrétní řádky.
\end{raw}

\subsection{Průběh iterací}

\begin{raw}
[RAW] Jednotlivé pilotní běhy. Pro každý běh:
\begin{itemize}
    \item Verze AGENTS.md (diff oproti předchozímu)
    \item Behavioral trace (co agent udělal)
    \item P/Q/E metriky
    \item Diagnostika: co selhalo a proč
    \item Změna instrukcí s odkazem na literaturu
\end{itemize}
\end{raw}

\section{Komparativní variace}
\label{sec:komparativni-vysledky}

\begin{raw}
[RAW] Systematické variace fungujících instrukcí.

\subsection{Ablace}

Pro každou ablaci:
\begin{itemize}
    \item Odebraná komponenta
    \item Hypotéza: co očekáváme
    \item P/Q/E metriky
    \item Pozorované chování vs. baseline
\end{itemize}

\subsection{Substituce}

Pro každou substituci:
\begin{itemize}
    \item Nahrazená komponenta a alternativa
    \item Hypotéza
    \item P/Q/E metriky
    \item Srovnání variant
\end{itemize}
\end{raw}

\section{Souhrnné výsledky}

\begin{raw}
[RAW] Across-run srovnání všech běhů (pilot + variace).
Tabulka se všemi P/Q/E metrikami per run.
\end{raw}
