\chapter{Metodika}
\label{kap:metodika}

\section{Výběr projektu pro case study}

\begin{raw}
Práce se zaměřuje na řízení a scaffolding - jde více do šířky než do hloubky. Proto potřebujeme menší projekt, na kterém můžeme spustit více běhů s různými nastaveními scaffoldingu a měřit výsledky.

Pro experiment potřebujeme projekt který:
\begin{itemize}
    \item \textbf{Hard logic} - jasná business pravidla, ne subjektivní výstupy (např. generování textu)
    \item \textbf{Jasné invarianty} - deterministické chování, matematicky ověřitelné správnost
    \item \textbf{Testovatelné} - lze objektivně měřit kvalitu výstupu
    \item \textbf{Přiměřená velikost} - menší projekt umožňuje více experimentálních běhů
    \item \textbf{Reálný use case} - prakticky využitelné, ne umělý příklad
\end{itemize}
\end{raw}

\subsection*{Systém upomínek faktur}

\begin{raw}
Systém pro automatické odesílání připomínek k nezaplaceným fakturám. Obsahuje:
\begin{itemize}
    \item Stavový automat pro sledování stavu faktury (nová, po splatnosti, upomínaná, eskalovaná)
    \item Časové výpočty (pracovní dny, ochranné lhůty)
    \item Pravidla pro eskalaci (kdy poslat další upomínku, kdy předat k vymáhání)
    \item Plánování odesílání upomínek
\end{itemize}
\end{raw}

\section{Zvolená metodika: Spec-Driven Development}

\begin{raw}
Pro referenční implementaci i experimenty je zvolena metodika \textbf{Spec-Driven Development (SDD)}
na úrovni \textbf{spec-first} \cite{sdd2026}.

\textbf{Zdůvodnění volby:}
\begin{enumerate}
    \item \textbf{Specifikace řídí implementaci} -- kvalita specifikace přímo ovlivňuje kvalitu výstupu agenta (viz 2.4.2 bod 7). SDD formalizuje tento princip.
    \item \textbf{Waterfall per increment} -- každý issue = jeden increment s detailní specifikací, ale mezi incrementy iterativní přístup. Odpovídá micro-waterfall hypotéze (2.3.4) podpořené empirickými daty \cite{watanabe2025agentprs, ehsani2026failedprs}.
    \item \textbf{Malé, focused úkoly} -- empirická data ukazují že malé agent PRs mají vyšší úspěšnost \cite{ehsani2026failedprs}. SDD spec-first přirozeně vede k dekompozici na testovatelné incrementy.
    \item \textbf{Spec-first stačí} -- pro jednorázový experiment není potřeba spec-anchored (udržovat sync spec-kód). Spec se napíše, agent implementuje, vyhodnotí se.
\end{enumerate}

\textbf{SDD workflow v kontextu BP:}
\begin{enumerate}
    \item \textbf{Specify} -- napsat GitHub Issue se strukturovanou specifikací (šablona viz níže)
    \item \textbf{Plan} -- (pro agenty: agent sám plánuje; pro referenci: autor plánuje)
    \item \textbf{Implement} -- implementace podle specifikace
    \item \textbf{Validate} -- testy (unit, acceptance criteria), review
\end{enumerate}
\end{raw}

\section{Referenční implementace}

\begin{raw}
Vlastní vývoj systému upomínek faktur se všemi náležitostmi softwarového inženýrství:
\begin{itemize}
    \item Specifikace a dokumentace
    \item Implementace (TypeScript)
    \item Testy (unit, integration)
    \item Git workflow (issues, commits, PR conventions)
    \item Quality gates (linting, type checking)
\end{itemize}

Tato implementace slouží jako "ground truth" pro porovnání.

\rule{\textwidth}{0.4pt}

\textbf{Formát specifikace: GitHub Issues}

Specifikace referenční implementace je strukturována jako GitHub Issues. Volba tohoto formátu vychází z:

\begin{enumerate}
    \item \textbf{Akademický standard} -- SWE-bench \cite{swebench2024}, de facto benchmark pro AI coding agenty (ICLR 2024), používá GitHub Issues jako specifikaci. 2294 úloh z reálných repozitářů.
    \item \textbf{Agilní RE praxe} -- v agilních týmech user stories a backlog items nahrazují formální SRS dokumenty \cite{cao2008}.
    \item \textbf{Open source praxe} -- issue trackery fungují jako de facto requirements management \cite{scacchi2002}.
    \item \textbf{Nativní čitelnost pro agenty} -- agent čte issues přes GitHub API nebo CLI, propojuje je s branches a PR.
    \item \textbf{Traceability} -- Issue \#N $\rightarrow$ branch $\rightarrow$ commits $\rightarrow$ PR $\rightarrow$ merge. Přirozená provázanost specifikace s implementací \cite{gotel1994}.
\end{enumerate}

Struktura každého issue vychází z empirického výzkumu o optimální specifikaci pro LLM agenty
(viz sekce 2.4.2, bod 7). Studie ukazují, že kvalita requirements přímo koreluje s kvalitou
LLM výstupu \cite{rope2024} a že tradiční user stories jsou příliš abstraktní pro přímý
vstup do LLM \cite{ullrich2025} -- je nutná dekompozice a obohacení o konkrétní kontext.

\textbf{Tři vrstvy specifikace:}

Obsah issue pokrývá tři úrovně abstrakce, které odpovídají standardní posloupnosti
v softwarovém inženýrství \cite[kap.~2]{sommerville2016}:

\begin{enumerate}
    \item \textbf{Requirements} (problémová doména -- CO business potřebuje):
    \begin{itemize}
        \item Title, Description -- účel a kontext funkcionality
        \item Acceptance criteria -- Given/When/Then, pozorovatelné chování \cite{ticoder2024}.
              Pokud jsou formulována s konkrétními hodnotami (vstupy, výstupy, stavy),
              jsou přímo mapovatelná na unit testy -- explicitní test cases tedy
              nejsou nutnou součástí specifikace, ale odvozitelným artefaktem
        \item Domain glossary -- sdílený slovník z business domény \cite{domaincodegen2024}
    \end{itemize}

    \item \textbf{Specification} (řešení -- CO přesně systém dělá):
    \begin{itemize}
        \item Inputs / Outputs -- datové typy, formáty, struktury \cite{wen2024io, specine2025}
        \item Preconditions / Postconditions -- stavové podmínky, invarianty \cite{newcomb2025prepost}
    \end{itemize}

    \item \textbf{Architecture} (struktura -- JAK je řešení organizované):
    \begin{itemize}
        \item Behavioral model -- state diagram, sekvenční logika \cite[kap.~5.4]{sommerville2016}
        \item Technické constraints -- tech stack, patterns, rozhraní
    \end{itemize}
\end{enumerate}

Toto rozdělení slouží i jako základ pro experimentální dimenzi ``úroveň detailu specifikace'' (viz sekce Experimenty):
referenční implementace používá plnou specifikaci (všechny tři vrstvy),
experimenty mohou poskytovat agentovi pouze vybrané vrstvy.

\textbf{Překryv elementů a problém redundance:}

Jednotlivé elementy šablony se částečně překrývají -- acceptance criteria implicitně
obsahují vstupy/výstupy (Given/Then), pre/postconditions (Given = precondition,
Then = postcondition) i behavioral model (přechody mezi stavy). Tato redundance
představuje problém: pro člověka vyšší cognitive overload, pro LLM agenta
plýtvání vzácným context window duplicitními informacemi.

Anthropic \cite{anthropic2025context} zavádí pojem \textbf{context rot} --
s rostoucím počtem tokenů klesá schopnost modelu přesně vzpomínat informace.
Doporučuje ``nejmenší možnou sadu high-signal tokenů''. IEEE 830 \cite{ieee830}
upozorňuje, že ``redundance sama o sobě není chyba, ale snadno k chybám vede''.
Bockeler \cite{bockeler2025sdd} kritizuje spec-kit (GitHub) za to, že specifikační
soubory jsou ``repetitive, both with each other, and with the code'' --
označuje to jako \textit{Verschlimmbesserung} (zhoršení snahou o zlepšení).

\textbf{Dvě publikum, různé potřeby:}

Specifikace slouží dvěma publikům současně: \textbf{AI agentovi} (implementuje
z~ní kód) a \textbf{lidskému vývojáři} (rozumí co se staví a kontroluje
co agent vytvořil). Kruchtenův 4+1 model \cite{kruchten1995} argumentuje,
že více pohledů je komplementárních \textbf{pro různá publika}.
Diagramy jsou pro člověka ``high-bandwidth'' komunikace (rychlé pochopení
celkové struktury), zatímco LLM zpracovávají Mermaid diagramy jako text.
Konkrétní Given/When/Then scénáře mohou být pro agenta účinnější
než vizuální model, ale pro člověka méně přehledné u~komplexních systémů.

Experimenty mohou ukázat optimální kombinaci elementů --
ne ``čím víc, tím lépe'', ale \textbf{která minimální sada}
reprezentací je efektivní pro obě publika současně.

\textbf{Empirické pořadí důležitosti:}

Studie Specine \cite{specine2025} empiricky měřila dopad jednotlivých elementů
na kvalitu generovaného kódu (Pass@1, 4 LLM, 5 benchmarků):

\textit{Tier 1 -- nejvyšší dopad:}
\begin{itemize}
    \item Příklady s vysvětlením ($\sim$14.5\%) $\rightarrow$ Acceptance criteria
    \item Účel specifikace ($\sim$13.5\%) $\rightarrow$ Description
    \item Výstupní požadavky ($\sim$11.6\%) $\rightarrow$ Outputs
\end{itemize}

\textit{Tier 2 -- silně doporučené:}
vstupní požadavky, klíčové pojmy, edge/corner cases.

\textit{Tier 3 -- hodnotné pro složité úlohy:}
pre/postconditions \cite{newcomb2025prepost}, error handling, behavioral model.

Tato šablona kombinuje přístupy podložené výzkumem:
structured natural language \cite[kap.~4.4]{sommerville2016}, test-driven specifikaci \cite{ticoder2024},
doménový kontext \cite{domaincodegen2024}, redukci specification misalignment \cite{specine2025},
design constraints \cite{newcomb2025prepost} a klarifikaci ambiguity \cite{clarifygpt2024}.
\end{raw}

\section{Experimenty}

\begin{raw}
Příprava různých nastavení scaffoldingu pro agenty:
\begin{itemize}
    \item Instrukce a procesy (jak má agent postupovat)
    \item Git automatizace a skripty
    \item Kontext který má agent k dispozici a který si vytváří
    \item Nastavení odvozená z literatury a spec-driven development přístupů
\end{itemize}

\textbf{Scaffolding jako experimentální dimenze:}

Kromě úrovně detailu specifikace je další dimenzí míra scaffoldingu v repozitáři.
Instrukce pro agenta jsou definovány v souboru \texttt{agents.md} (emerging standard
pro AI coding agenty), který obsahuje konvence pro git workflow, testování,
code quality a proces vývoje. Experimenty mohou testovat i schopnost agenta
nastavit si scaffolding sám (linting, test framework, project structure) vs.
dostat ho předpřipravený.

\textbf{Klíčová experimentální dimenze -- úroveň detailu specifikace:}

Referenční implementace používá plnou specifikaci (requirements + specification + architecture).
Experimenty poskytují agentovi různé podmnožiny:

\begin{itemize}
    \item \textbf{Full} -- všechny tři vrstvy (requirements + specification + architecture)
    \item \textbf{Medium} -- requirements + specification (bez architektonických rozhodnutí)
    \item \textbf{Minimal} -- pouze requirements (Title, Description, Acceptance criteria, Glossary)
\end{itemize}

Toto měří dopad úrovně specifikace na kvalitu výstupu agenta -- empiricky
podpořeno studiemi \cite{specine2025, rope2024, yang2025underspec}.

Spuštění agentů s různými nastaveními scaffoldingu na stejném zadání. Zaznamenání průběhu a výstupů.

\textbf{Odvozování testů z acceptance criteria:}

Acceptance criteria ve formátu Given/When/Then (BDD) s konkrétními hodnotami
jsou přímo mapovatelná na unit testy. Schopnost agenta korektně odvodit test suite
z~acceptance criteria je měřitelná dimenze experimentu -- odpovídá zjištění
TiCoder \cite{ticoder2024}, kde formalizace záměru přes testy vedla k~45.97\%
zlepšení Pass@1.
\end{raw}

\section{Analýza}

\begin{raw}
[RAW]
Tradiční přístupy k měření kvality SW (podpora pro volbu dimenzí):

\textbf{Sommerville (Ch. 24, s. 705--728):}
\begin{itemize}
    \item Rozlišuje \textbf{control metrics} (procesní -- sledují proces vývoje) vs. \textbf{predictor metrics} (produktové -- měří vlastnosti kódu/dokumentů)
    \item Vztah proces-produkt u SW není přímočarý jako ve výrobě -- SW je designován, ne vyráběn, vliv individuálních dovedností je velký (s. 706)
    \item Produktové metriky (LOC, cyklomatická složitost) nemají jasný a konzistentní vztah ke kvalitativním atributům (s. 721)
    \item → Naše dimenze Functional Quality = predictor metrics, Compliance = control metrics
\end{itemize}

\textbf{McConnell -- Code Complete (Ch. 28, s. 715, Table 28-2):}
\begin{itemize}
    \item Kategorie měření: Size (LOC, třídy, komentáře) a Overall Quality (počet defektů, defekty/KLOC, mean time between failures)
    \item Praktický pohled -- co se dá reálně měřit v projektu
\end{itemize}

\textbf{SWEBOK v4 (Ch. 12, s. 248--256; Ch. 6, s. 176):}
\begin{itemize}
    \item Software Quality Measurement (s. 253) -- kvantifikace atributů pro rozhodování
    \item Míry údržby (s. 176): complexity, maintainability, testability, supportability, reliability
    \item Odkaz na ISO 25010 jako standard pro kvalitativní charakteristiky (s. 46, 256)
\end{itemize}

\textbf{SWE-bench (Appendix C.7, s. 28):}
\begin{itemize}
    \item Cyklomatická složitost (McCabe) a Halstead measures jako metriky pro hodnocení kódu v benchmarku
    \item Příklad jak existující benchmarky měří kvalitu kódu agentů -- ale jen funkční/strukturální, ne procesní
\end{itemize}

\textbf{Jin et al. 2024 -- LLM Agents SWE Survey (Table VII, s. 21):}
\begin{itemize}
    \item Přehled evaluačních metrik: Accuracy, Pass@k, Task Completion Time, Task Success, Execution Accuracy, Win-Rate
    \item Většina existující literatury měří hlavně funkční kvalitu výstupu
    \item → Naše dimenze Compliance a Alignment jsou méně pokryté v literatuře -- vlastní přínos
\end{itemize}
\end{raw}

\begin{raw}
Hodnocení výstupů agentů probíhá ve čtyřech dimenzích: funkční kvalita, procesní kvalita, efektivita a alignment.

\subsection{Functional Quality (Funkční kvalita)}

Měření funkčních vlastností výstupu dle ISO 25010:

\textbf{Completeness (Úplnost):}
\begin{itemize}
    \item Míra pokrytí požadované funkcionality
    \item Měření: Checklist požadavků ze specifikace → procento implementovaných
\end{itemize}

\textbf{Correctness (Správnost):}
\begin{itemize}
    \item Správnost implementace - funguje to jak má?
    \item Měření: Spuštění referenčních testů na kód agenta (pass rate)
    \item Kvalita testů agenta: Mutation testing (Stryker) - mutation score určuje jak dobře testy detekují chyby
\end{itemize}

\subsection{Compliance (Procesní kvalita)}

Dodržování softwarově-inženýrských praktik:

\textbf{Workflow:}
\begin{itemize}
    \item Dodržení flow: issues → branch → commits → PR
    \item Měření: Automatická kontrola git historie a GitHub artefaktů
\end{itemize}

\textbf{Conventions (Konvence):}
\begin{itemize}
    \item Kvalita commit messages (formát, atomicita, srozumitelnost)
    \item Kvalita issues (popis, acceptance criteria)
    \item Kvalita dokumentace a PR description
    \item Měření: LLM-as-a-judge s definovaným rubrikem \cite{llmjudge2024}
\end{itemize}

\textbf{Transparency (Transparentnost):}
\begin{itemize}
    \item Vysvětluje agent svá rozhodnutí?
    \item Dokumentuje postup a důvody?
    \item Měření: LLM-as-a-judge + manuální review
\end{itemize}

\subsection{Efficiency (Efektivita)}

Náklady na dosažení výsledku:

\begin{itemize}
    \item \textbf{Token usage} - spotřeba tokenů (náklady na API)
    \item \textbf{Iterations} - počet pokusů a oprav potřebných k dokončení
    \item \textbf{Time} - celkový čas do dokončení
    \item \textbf{Human intervention} - míra nutných lidských zásahů a korekcí
\end{itemize}

Měření: Logování z agenta a konverzačních sessions.

\subsection{Metody měření}

Kombinace tří přístupů:
\begin{itemize}
    \item \textbf{Automatické} - testy, mutation testing, git log analýza, token counting
    \item \textbf{LLM-as-a-judge} - hodnocení subjektivních aspektů (kvalita commit messages, dokumentace) pomocí LLM s definovaným rubrikem
    \item \textbf{Manuální review} - kvalitativní zhodnocení celku autorem
\end{itemize}

LLM-as-a-judge přístup využívá strukturované hodnocení kde LLM dostane kritéria a škálu, a konzistentně hodnotí všechny běhy. Validace tohoto přístupu probíhá porovnáním s manuálním hodnocením na vzorku \cite{llmjudge2024}.

\subsection{Alignment (Soulad se záměrem)}

Alignment měří, zda agent pochopil skutečný záměr zadání - ne jen doslovnou instrukci, ale co uživatel skutečně chtěl \cite{llmjudge2024}.

Agent může mít 100\% Correctness a Completeness, ale být misaligned - technicky splnil zadání, ale výsledek neodpovídá záměru.

\textbf{Co se hodnotí:}
\begin{itemize}
    \item \textbf{Over-engineering} - přidal agent funkcionalitu která nebyla požadována?
    \item \textbf{Under-delivering} - vynechal agent implicitní požadavky které byly zřejmé z kontextu?
    \item \textbf{Misinterpretation} - pochopil agent zadání špatně?
    \item \textbf{Scope adherence} - držel se agent vymezeného rozsahu?
\end{itemize}

\textbf{Měření:} Manuální review autorem + LLM-as-a-judge porovnávající zadání vs. skutečný výstup.

\subsection{Vyhodnocení}

Identifikace vzorů - která nastavení scaffoldingu vedla k lepším výsledkům v jednotlivých dimenzích. Porovnání trade-offs (např. vyšší kvalita vs. vyšší náklady).
\end{raw}
