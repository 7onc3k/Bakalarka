\chapter{Metodika}
\label{kap:metodika}

\section{Výběr projektu pro case study}

\begin{raw}
Práce se zaměřuje na řízení a scaffolding - jde více do šířky než do hloubky. Proto potřebujeme menší projekt, na kterém můžeme spustit více běhů s různými nastaveními scaffoldingu a měřit výsledky.

Pro experiment potřebujeme projekt který:
\begin{itemize}
    \item \textbf{Hard logic} - jasná business pravidla, ne subjektivní výstupy (např. generování textu)
    \item \textbf{Jasné invarianty} - deterministické chování, matematicky ověřitelné správnost
    \item \textbf{Testovatelné} - lze objektivně měřit kvalitu výstupu
    \item \textbf{Přiměřená velikost} - menší projekt umožňuje více experimentálních běhů
    \item \textbf{Reálný use case} - prakticky využitelné, ne umělý příklad
\end{itemize}
\end{raw}

\subsection*{Systém upomínek faktur}

\begin{raw}
Systém pro automatické odesílání připomínek k nezaplaceným fakturám. Obsahuje:
\begin{itemize}
    \item Stavový automat pro sledování stavu faktury (nová, po splatnosti, upomínaná, eskalovaná)
    \item Časové výpočty (pracovní dny, ochranné lhůty)
    \item Pravidla pro eskalaci (kdy poslat další upomínku, kdy předat k vymáhání)
    \item Plánování odesílání upomínek
\end{itemize}
\end{raw}

\section{Referenční implementace}

\begin{raw}
Vlastní vývoj systému upomínek faktur se všemi náležitostmi softwarového inženýrství:
\begin{itemize}
    \item Specifikace a dokumentace
    \item Implementace (TypeScript)
    \item Testy (unit, integration)
    \item Git workflow (issues, commits, PR conventions)
    \item Quality gates (linting, type checking)
\end{itemize}

Tato implementace slouží jako "ground truth" pro porovnání.

\rule{\textwidth}{0.4pt}

\textbf{Formát specifikace: GitHub Issues}

Specifikace referenční implementace je strukturována jako GitHub Issues. Volba tohoto formátu vychází z:

\begin{enumerate}
    \item \textbf{Akademický standard} -- SWE-bench \cite{swebench2024}, de facto benchmark pro AI coding agenty (ICLR 2024), používá GitHub Issues jako specifikaci. 2294 úloh z reálných repozitářů.
    \item \textbf{Agilní RE praxe} -- v agilních týmech user stories a backlog items nahrazují formální SRS dokumenty \cite{cao2008}.
    \item \textbf{Open source praxe} -- issue trackery fungují jako de facto requirements management \cite{scacchi2002}.
    \item \textbf{Nativní čitelnost pro agenty} -- agent čte issues přes GitHub API nebo CLI, propojuje je s branches a PR.
    \item \textbf{Traceability} -- Issue \#N $\rightarrow$ branch $\rightarrow$ commits $\rightarrow$ PR $\rightarrow$ merge. Přirozená provázanost specifikace s implementací \cite{gotel1994}.
\end{enumerate}

Struktura každého issue vychází z empirického výzkumu o optimální specifikaci pro LLM agenty
(viz sekce 2.4.2, bod 7):

\textbf{Šablona issue:}
\begin{enumerate}
    \item \textbf{Title} -- stručný popis funkcionality
    \item \textbf{Description} -- co a proč (structured natural language \cite[kap.~4.4]{sommerville2016})
    \item \textbf{Domain glossary} -- klíčové pojmy z business domény \cite{domaincodegen2024}
    \item \textbf{Inputs / Outputs} -- konkrétní I/O příklady (redukuje misinterpretaci \cite{specine2025})
    \item \textbf{Preconditions / Postconditions} -- stavové podmínky
    \item \textbf{Acceptance criteria} -- Given/When/Then, mapovatelné na testy \cite{ticoder2024}
    \item \textbf{Behavioral model} -- state diagram (Mermaid) pro event-driven logiku \cite[kap.~5.4]{sommerville2016}
\end{enumerate}

Tato šablona kombinuje čtyři přístupy podložené výzkumem:
structured natural language (Sommerville), test-driven specifikaci (TiCoder),
doménový kontext (ACM TOSEM domain-specific study),
a redukci specification misalignment (Specine).
\end{raw}

\section{Experimenty}

\begin{raw}
Příprava různých nastavení scaffoldingu pro agenty:
\begin{itemize}
    \item Instrukce a procesy (jak má agent postupovat)
    \item Git automatizace a skripty
    \item Kontext který má agent k dispozici a který si vytváří
    \item Nastavení odvozená z literatury a spec-driven development přístupů
\end{itemize}

Spuštění agentů s různými nastaveními scaffoldingu na stejném zadání. Zaznamenání průběhu a výstupů.
\end{raw}

\section{Analýza}

\begin{raw}
Hodnocení výstupů agentů probíhá ve čtyřech dimenzích: funkční kvalita, procesní kvalita, efektivita a alignment.

\subsection{Functional Quality (Funkční kvalita)}

Měření funkčních vlastností výstupu dle ISO 25010:

\textbf{Completeness (Úplnost):}
\begin{itemize}
    \item Míra pokrytí požadované funkcionality
    \item Měření: Checklist požadavků ze specifikace → procento implementovaných
\end{itemize}

\textbf{Correctness (Správnost):}
\begin{itemize}
    \item Správnost implementace - funguje to jak má?
    \item Měření: Spuštění referenčních testů na kód agenta (pass rate)
    \item Kvalita testů agenta: Mutation testing (Stryker) - mutation score určuje jak dobře testy detekují chyby
\end{itemize}

\subsection{Compliance (Procesní kvalita)}

Dodržování softwarově-inženýrských praktik:

\textbf{Workflow:}
\begin{itemize}
    \item Dodržení flow: issues → branch → commits → PR
    \item Měření: Automatická kontrola git historie a GitHub artefaktů
\end{itemize}

\textbf{Conventions (Konvence):}
\begin{itemize}
    \item Kvalita commit messages (formát, atomicita, srozumitelnost)
    \item Kvalita issues (popis, acceptance criteria)
    \item Kvalita dokumentace a PR description
    \item Měření: LLM-as-a-judge s definovaným rubrikem \cite{llmjudge2024}
\end{itemize}

\textbf{Transparency (Transparentnost):}
\begin{itemize}
    \item Vysvětluje agent svá rozhodnutí?
    \item Dokumentuje postup a důvody?
    \item Měření: LLM-as-a-judge + manuální review
\end{itemize}

\subsection{Efficiency (Efektivita)}

Náklady na dosažení výsledku:

\begin{itemize}
    \item \textbf{Token usage} - spotřeba tokenů (náklady na API)
    \item \textbf{Iterations} - počet pokusů a oprav potřebných k dokončení
    \item \textbf{Time} - celkový čas do dokončení
    \item \textbf{Human intervention} - míra nutných lidských zásahů a korekcí
\end{itemize}

Měření: Logování z agenta a konverzačních sessions.

\subsection{Metody měření}

Kombinace tří přístupů:
\begin{itemize}
    \item \textbf{Automatické} - testy, mutation testing, git log analýza, token counting
    \item \textbf{LLM-as-a-judge} - hodnocení subjektivních aspektů (kvalita commit messages, dokumentace) pomocí LLM s definovaným rubrikem
    \item \textbf{Manuální review} - kvalitativní zhodnocení celku autorem
\end{itemize}

LLM-as-a-judge přístup využívá strukturované hodnocení kde LLM dostane kritéria a škálu, a konzistentně hodnotí všechny běhy. Validace tohoto přístupu probíhá porovnáním s manuálním hodnocením na vzorku \cite{llmjudge2024}.

\subsection{Alignment (Soulad se záměrem)}

Alignment měří, zda agent pochopil skutečný záměr zadání - ne jen doslovnou instrukci, ale co uživatel skutečně chtěl \cite{llmjudge2024}.

Agent může mít 100\% Correctness a Completeness, ale být misaligned - technicky splnil zadání, ale výsledek neodpovídá záměru.

\textbf{Co se hodnotí:}
\begin{itemize}
    \item \textbf{Over-engineering} - přidal agent funkcionalitu která nebyla požadována?
    \item \textbf{Under-delivering} - vynechal agent implicitní požadavky které byly zřejmé z kontextu?
    \item \textbf{Misinterpretation} - pochopil agent zadání špatně?
    \item \textbf{Scope adherence} - držel se agent vymezeného rozsahu?
\end{itemize}

\textbf{Měření:} Manuální review autorem + LLM-as-a-judge porovnávající zadání vs. skutečný výstup.

\subsection{Vyhodnocení}

Identifikace vzorů - která nastavení scaffoldingu vedla k lepším výsledkům v jednotlivých dimenzích. Porovnání trade-offs (např. vyšší kvalita vs. vyšší náklady).
\end{raw}
