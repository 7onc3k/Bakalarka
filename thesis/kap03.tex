\chapter{Metodika}
\label{kap:metodika}

\section{Výběr projektu pro case study}

\begin{raw}
Práce se zaměřuje na řízení a scaffolding - jde více do šířky než do hloubky. Proto potřebujeme menší projekt, na kterém můžeme spustit více běhů s různými nastaveními scaffoldingu a měřit výsledky.

Pro experiment potřebujeme projekt který:
\begin{itemize}
    \item \textbf{Hard logic} - jasná business pravidla, ne subjektivní výstupy (např. generování textu)
    \item \textbf{Jasné invarianty} - deterministické chování, matematicky ověřitelné správnost
    \item \textbf{Testovatelné} - lze objektivně měřit kvalitu výstupu
    \item \textbf{Přiměřená velikost} - menší projekt umožňuje více experimentálních běhů
    \item \textbf{Reálný use case} - prakticky využitelné, ne umělý příklad
\end{itemize}
\end{raw}

\subsection*{Systém upomínek faktur}

\begin{raw}
Systém pro automatické odesílání připomínek k nezaplaceným fakturám. Obsahuje:
\begin{itemize}
    \item Stavový automat pro sledování stavu faktury (nová, po splatnosti, upomínaná, eskalovaná)
    \item Časové výpočty (pracovní dny, ochranné lhůty)
    \item Pravidla pro eskalaci (kdy poslat další upomínku, kdy předat k vymáhání)
    \item Plánování odesílání upomínek
\end{itemize}
\end{raw}

\section{Referenční implementace}

\begin{raw}
Vlastní vývoj systému upomínek faktur se všemi náležitostmi softwarového inženýrství:
\begin{itemize}
    \item Specifikace a dokumentace
    \item Implementace (TypeScript)
    \item Testy (unit, integration)
    \item Git workflow (issues, commits, PR conventions)
    \item Quality gates (linting, type checking)
\end{itemize}

Tato implementace slouží jako "ground truth" pro porovnání.
\end{raw}

\section{Experimenty}

\begin{raw}
Příprava různých nastavení scaffoldingu pro agenty:
\begin{itemize}
    \item Instrukce a procesy (jak má agent postupovat)
    \item Git automatizace a skripty
    \item Kontext který má agent k dispozici a který si vytváří
    \item Nastavení odvozená z literatury a spec-driven development přístupů
\end{itemize}

Spuštění agentů s různými nastaveními scaffoldingu na stejném zadání. Zaznamenání průběhu a výstupů.
\end{raw}

\section{Analýza}

\begin{raw}
Měření výstupů proti referenční implementaci:
\begin{itemize}
    \item \textbf{Functional Quality} (dle ISO 25010):
    \begin{itemize}
        \item Completeness - míra pokrytí požadované funkcionality
        \item Correctness - správnost výsledků
    \end{itemize}
    \item \textbf{Compliance} (dodržování procesů):
    \begin{itemize}
        \item Workflow (issues → branch → PR)
        \item Konvence (commit messages, dokumentace)
        \item Transparentnost (vysvětluje co dělá a proč)
    \end{itemize}
\end{itemize}

Identifikace vzorů - která nastavení scaffoldingu vedla k lepším výsledkům.
\end{raw}

\begin{raw}
Poznámka: Konkrétní metriky pro hodnocení výstupu agentů jsou aktivní oblast výzkumu. V rámci práce bude provedena dodatečná rešerše dostupných akademických přístupů.
\end{raw}
