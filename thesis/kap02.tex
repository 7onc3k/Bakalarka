\chapter{Teoretická východiska}
\label{kap:teorie}

\begin{draft}
Tato kapitola vysvětluje, co je softwarové inženýrství, proč vzniklo a jaké jsou jeho současné trendy. Nejprve je popsáno tradiční softwarové inženýrství, následně je zkoumáno, jak se oblast mění díky AI coding agentům, a nakonec jsou představeny prvky scaffoldingu (podpůrných struktur), které agenti využívají.
\end{draft}

\section{Tradiční SWE}

\begin{raw}
\textbf{Hlavní body sekce (co má čtenář pochopit):}
\begin{enumerate}
    \item \textbf{Proč SWE vzniklo} - software je komplexní, bez struktury to nefunguje
    \begin{itemize}
        \item NATO 1968 konference - "software crisis"
        \item Projekty selhávaly, potřeba disciplíny
    \end{itemize}

    \item \textbf{Proč procesy existují} - nejsou byrokracie, jsou odpověď na reálné problémy:
    \begin{itemize}
        \item Koordinace mezi lidmi
        \item Udržení konzistence
        \item Předávání znalostí
    \end{itemize}

    \item \textbf{Role lidí a komunikace} - klíčové pro kontext agentů:
    \begin{itemize}
        \item SWE je fundamentálně týmová disciplína
        \item Hodně znalostí je implicitních (v hlavách lidí, ne v dokumentaci)
        \item Brooks's Law - komunikační režie roste exponenciálně s velikostí týmu
    \end{itemize}

    \item \textbf{SDLC jako struktura} - fáze existují z důvodu
    \begin{itemize}
        \item requirements → design → implementation → testing → maintenance
        \item Každá fáze řeší specifické problémy
    \end{itemize}
\end{enumerate}

\textbf{Proč tyhle body:} Když pak v 2.2 řekneme "agenti nepotřebují meetingy ale vše musí být explicitní" - čtenář pochopí co se mění. Implicitní znalosti v týmu → explicitní instrukce pro agenta.

\textbf{Zdroje:}
\begin{itemize}
    \item SWEBOK v4 - referenční příručka pro terminologii a strukturu oboru
    \item NATO 1968 - historický kontext
    \item Brooks - Mythical Man-Month (Brooks's Law)
\end{itemize}
\end{raw}

\section{SWE s coding agents}

\begin{raw}
\textbf{Co sem patří:}
\begin{itemize}
    \item Jak agenti mění vývoj
    \item Nástroje pro řízení agentů (agents.md, skills.md, CLAUDE.md, hooks)
    \item Co je stejné, co je jiné oproti tradičnímu SWE
    \item Zdroje: surveys o LLM agentech v SE (máme v literatuře)
\end{itemize}

\textbf{Poznámka - trade-off agenti vs týmy:}
\begin{itemize}
    \item Tradiční SWE = hodně o organizaci a lidech (komunikace, koordinace, předávání znalostí)
    \item Velké týmy = komunikační režie (Brooks's Law)
    \item Agenti nepotřebují meetingy, ale nerozumí nuancím - vše musí být explicitní
    \item Žádné implicitní porozumění z konverzace ("řekli jsme si na meetingu")
    \item Hypotéza: malé týmy s agenty mohou růst rychleji než velké týmy bez nich
\end{itemize}
\end{raw}

\section{Scaffolding pro agenty}

\begin{raw}
\textbf{Co sem patří:}
\begin{itemize}
    \item Co je scaffolding (definice)
    \item Jak propojit nástroje s agenty
    \item Co agent potřebuje aby fungoval dobře
    \item Spec-driven development (BMAD, SpecKit...)
    \item Metriky pro hodnocení agentů
\end{itemize}

\textbf{Poznámky k zapracování:}
\begin{itemize}
    \item ITIL/CMMI relevance - ověřit jestli vůbec patří do BP
    \item Viz handoffs/03-agent-framework-brainstorm.md a 06-agent-framework-consolidated.md
\end{itemize}
\end{raw}
