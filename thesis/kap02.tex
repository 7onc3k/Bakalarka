\chapter{Teoretická východiska}
\label{kap:teorie}

% === ÚVOD KAPITOLY (mezi 2 a 2.1) - ROADMAP ===
\begin{draft}
Tato kapitola vysvětluje teoretická východiska práce. Nejprve je popsáno softwarové inženýrství jako disciplína, následně životní cyklus a metodiky vývoje software. Poté je zkoumáno, jak se oblast mění díky AI coding agentům, a nakonec jsou představeny prvky scaffoldingu (podpůrných struktur), které agenti využívají.
\end{draft}

%%%%%%%%%%%%%%%%%%%%%%%%%%%%%%%%%%%%%%%%%%%%%%%%%%%%%%%%%%%%%%%%%%%%%%%%%%%%%%%
\section{Softwarové inženýrství}
%%%%%%%%%%%%%%%%%%%%%%%%%%%%%%%%%%%%%%%%%%%%%%%%%%%%%%%%%%%%%%%%%%%%%%%%%%%%%%%

\begin{draft}
Pro pochopení toho, jak AI agenti mění vývoj software, je nezbytné nejprve porozumět zavedeným postupům a standardům softwarového inženýrství. Tato sekce definuje softwarové inženýrství, vysvětluje proč je software inherentně složitý a jak tato složitost vedla ke vzniku oboru.
\end{draft}

% === PODSEKCE 2.1.1 ===
\subsection{Definice a vymezení oboru}

\begin{draft}
Softwarové inženýrství je disciplína, která se zabývá celým životním cyklem software -- od specifikace až po údržbu \cite[s.~xxxvii]{swebok2024}.

Na rozdíl od programování, které se soustředí na implementaci a technické aspekty jako algoritmy a datové struktury, softwarové inženýrství přistupuje k vývoji software holisticky -- zahrnuje nejen technickou stránku, ale i organizační aspekty jako řízení projektů a rozpočty \cite[s.~21]{sommerville2016}.
\end{draft}

\begin{raw}
\textbf{TODO:} Rozšířit definice a vymezení oboru (1-2 odstavce).
\end{raw}

% === PODSEKCE 2.1.2 ===
\subsection{Historický kontext}

\begin{draft}
Systémy jako software jsou čím dál složitější \cite[s.~582]{sommerville2016}.
\end{draft}

\begin{raw}
\textbf{Narativní flow (revidovaný):}
\begin{enumerate}
    \item \textbf{Složitost systémů roste} -- software je čím dál komplexnější
          \\ → \texttt{sommerville2016} s.~582:
          \\ \textit{``The root cause of these problems is, as it was in the 1960s, that we are trying to build systems that are larger and more complex than before. We are attempting to build these `mega-systems' using methods and technology that were never designed for this purpose.''}

    \item \textbf{Abstrakce jako nástroj} -- reakce na složitost:
    \begin{itemize}
        \item Assembler → C → Java → frameworky
        \item Každá vrstva skrývá detaily (information hiding)
              \\ → \texttt{colburn2000} s.~1:
              \\ \textit{``Abstraction through information hiding is a primary factor in computer science progress and success through an examination of the ubiquitous role of information hiding in programming languages, operating systems, network architecture, and design patterns.''}
        \item Dijkstra: abstrakce pomáhá být přesný, ne vágní
              \\ → \texttt{swebok2024} s.~370:
              \\ \textit{``Dijkstra states: `The purpose of abstracting is not to be vague, but to create a new semantic level in which one can be absolutely precise.'''}
    \end{itemize}

    \item \textbf{Přesto 1968 krize} -- proč?
    \begin{itemize}
        \item Složitost rostla rychleji než naše schopnost ji zvládat
        \item NATO konference: ``software crisis''
              \\ → \texttt{nato1968} s.~78 (Perlis keynote):
              \\ \textit{``I believe it is because we recognize that a practical problem of considerable difficulty and importance has arisen: The successful design, production and maintenance of useful software systems.''}
    \end{itemize}

    \item \textbf{Reakce: vznik SWE} -- disciplína pro řízení složitosti
          \\ → \texttt{sommerville2016} s.~592:
          \\ \textit{``In software engineering, we have seen the incredibly rapid development of the discipline to help manage the increasing size and complexity of software systems... the approach that has been the basis of complexity management in software engineering is called reductionism.''}

    \item \textbf{Dnes: AI jako nový problém}
    \begin{itemize}
        \item Není to jen další vrstva abstrakce
        \item Je to ztráta determinismu -- ``black box''
        \item Proto mechanistic interpretability
              \\ → \texttt{bereska2024} s.~1:
              \\ \textit{``Understanding AI systems' inner workings is critical for ensuring value alignment and safety. This review explores mechanistic interpretability: reverse engineering the computational mechanisms and representations learned by neural networks into human-understandable algorithms and concepts to provide a granular, causal understanding.''}
    \end{itemize}
\end{enumerate}

\textbf{Klíčový insight:}
Abstrakce $\neq$ složitost. Abstrakce je NÁSTROJ pro zvládání složitosti.
AI přináší kvalitativně nový problém (nedeterminismus), ne jen ``více abstrakce''.
\end{raw}

% === PODSEKCE 2.1.3 ===
\subsection{Komplexita software}

\begin{raw}
\textbf{Co sem patří (Brooks -- No Silver Bullet):}
\begin{itemize}
    \item \textbf{Essential complexity} -- inherentní složitost problému
          \\ → \texttt{brooks1987} s.~6:
          \\ \textit{``The complexity of software is an essential property, not an accidental one. Hence descriptions of a software entity that abstract away its complexity often abstract away its essence.''}
    \begin{itemize}
        \item Business logika, requirements, doménová znalost
        \item Nelze odstranit -- je to podstata toho co řešíme
    \end{itemize}

    \item \textbf{Accidental complexity} -- složitost kterou si přidáváme
    \begin{itemize}
        \item Nástroje, technologie, jazyky, frameworky
        \item Lze redukovat lepšími nástroji a abstrakcemi
    \end{itemize}

    \item \textbf{Vlastnosti software} -- proč je jiný než fyzické systémy:
          \\ → \texttt{brooks1987} s.~6:
          \\ \textit{``Software entities are more complex for their size than perhaps any other human construct, because no two parts are alike.''}
    \begin{itemize}
        \item Neviditelný, snadno měnitelný, nemá fyzická omezení
        \item \textbf{Nelineární růst komplexity:} Na rozdíl od fyzických systémů
              (např. stavba zdi -- cihla na cihlu, proces stále stejný),
              u software každý nový prvek interaguje s ostatními a přidává nové stavy.
              \\ → \texttt{brooks1987} s.~6:
              \\ \textit{``A scaling-up of a software entity is not merely a repetition of the same elements in larger size, it is necessarily an increase in the number of different elements. In most cases, the elements interact with each other in some non-linear fashion, and the complexity of the whole increases much more than linearly.''}
        \item → \texttt{mcconnell2004} s.~127:
              \\ \textit{``People use abstraction continuously. If you had to deal with individual wood fibers, varnish molecules, and steel molecules every time you used your front door, you'd hardly make it in or out of your house each day. Abstraction is a big part of how we deal with complexity in the real world.''}
    \end{itemize}

    \item \textbf{Evoluce a růst komplexity v čase} (Lehman's Laws):
          \\ → \texttt{lehman1980} s.~9:
          \\ \textit{``I. Continuing Change -- A program that is used... undergoes continual change or becomes progressively less useful. The change or decay process continues until it is judged more cost effective to replace the system with a recreated version.''}
          \\ \textit{``II. Increasing Complexity -- As an evolving program is continually changed, its complexity increases unless work is done to maintain or reduce it.''}
    \begin{itemize}
        \item Software musí být neustále adaptován (zákon I)
        \item Komplexita roste s časem pokud se aktivně neredukuje (zákon II)
        \item Doplňuje Brookse: Brooks říká PROČ je software složitý, Lehman říká že komplexita navíc ROSTE
    \end{itemize}
\end{itemize}

\textbf{Propojení s 2.1.2:}
Abstrakce (kompilátory, frameworky) řeší accidental complexity -- ale essential zůstává. To je důvod proč ``no silver bullet''.
\end{raw}

%%%%%%%%%%%%%%%%%%%%%%%%%%%%%%%%%%%%%%%%%%%%%%%%%%%%%%%%%%%%%%%%%%%%%%%%%%%%%%%
\section{Životní cyklus a metodiky}
%%%%%%%%%%%%%%%%%%%%%%%%%%%%%%%%%%%%%%%%%%%%%%%%%%%%%%%%%%%%%%%%%%%%%%%%%%%%%%%

\begin{raw}
\textbf{Úvod sekce 2.2 (1-2 věty):}
Tato sekce popisuje jak se software vyvíjí v praxi: co se dělá (fáze), jak se to organizuje (metodiky), čím se to dělá (nástroje), co vzniká (artefakty), a kdo to dělá (role a komunikace).
\end{raw}

% === PODSEKCE 2.2.1 ===
\subsection{Fáze životního cyklu}

\begin{raw}
\textbf{Co sem patří:}
\begin{itemize}
    \item Sommerville - 4 základní aktivity:
    \begin{enumerate}
        \item Software specification
        \item Software development
        \item Software validation
        \item Software evolution
    \end{enumerate}
    \item Každá fáze produkuje artefakty
    \item Cyklický charakter - software se neustále vyvíjí
\end{itemize}

\textbf{Zdroje:}
\begin{itemize}
    \item Sommerville - Software Engineering (SDLC)
    \item SWEBOK v4 - Software Engineering Process KA
\end{itemize}
\end{raw}

% === PODSEKCE 2.2.2 ===
\subsection{Modely a metodiky}

\begin{raw}
\textbf{Co sem patří:}
\begin{itemize}
    \item Waterfall - sekvenční, dokumentace dopředu
    \item Iterativní/inkrementální - opakované cykly
    \item Agile - flexibilita, rychlý feedback, spolupráce
    \item Různé modely = různé způsoby organizace stejných fází
    \item Trade-offs: prediktabilita vs flexibilita
\end{itemize}

\textbf{Zdroje:}
\begin{itemize}
    \item Sommerville - process models
    \item Boehm - Spiral Model (1988)
    \item Beck - Extreme Programming Explained
    \item Agile Manifesto
\end{itemize}
\end{raw}

% === PODSEKCE 2.2.3 ===
\subsection{Nástroje}

\begin{raw}
\textbf{Co sem patří:}
\begin{itemize}
    \item IDE - integrovaná vývojová prostředí
    \item Version control (Git) - správa verzí, spolupráce
    \item CI/CD - automatizace buildů, testů, deploymentu
    \item Issue tracking, dokumentace
    \item Nástroje jako součást procesů - umožňují škálování
\end{itemize}

\textbf{Proč je to důležité pro BP:}
Agenti využívají tyto nástroje - musí s nimi umět pracovat. Scaffolding staví na existujících nástrojích.

\textbf{Zdroje:}
\begin{itemize}
    \item SWEBOK v4 - Software Engineering Tools KA
    \item Praktické znalosti (Git, GitHub Actions, etc.)
\end{itemize}
\end{raw}

% === PODSEKCE 2.2.4 ===
\subsection{Artefakty}

\begin{raw}
\textbf{Co sem patří:}
\begin{itemize}
    \item Specifikace a požadavky (requirements)
    \item Návrhy a architektura (design docs)
    \item Zdrojový kód
    \item Testy (unit, integration, e2e)
    \item Dokumentace (technická, uživatelská)
    \item Artefakty jako vstup/výstup fází životního cyklu
\end{itemize}

\textbf{Proč je to důležité pro BP:}
Agenti produkují a konzumují tyto artefakty. Kvalita artefaktů ovlivňuje kvalitu výstupu agenta.

\textbf{Zdroje:}
\begin{itemize}
    \item Sommerville - Software Engineering
    \item SWEBOK v4
\end{itemize}
\end{raw}

% === PODSEKCE 2.2.5 ===
\subsection{Role a komunikace}

\begin{raw}
\textbf{Co sem patří:}
\begin{itemize}
    \item SWE je fundamentálně týmová disciplína
    \item Hodně znalostí je implicitních (v hlavách lidí, ne v dokumentaci)
    \item Brooks's Law - komunikační režie roste exponenciálně s velikostí týmu
    \item Procesy slouží ke koordinaci, konzistenci, předávání znalostí
    \item Nástroje a artefakty jako prostředky komunikace (CSCW perspektiva)
\end{itemize}

\textbf{Proč je to důležité pro BP:}
Když pak v 2.3 řekneme "agenti nepotřebují meetingy ale vše musí být explicitní" - čtenář pochopí co se mění. Implicitní znalosti v týmu → explicitní instrukce pro agenta.

\textbf{Zdroje:}
\begin{itemize}
    \item Brooks - Mythical Man-Month (Brooks's Law)
    \item Beck - Extreme Programming Explained (týmová spolupráce)
    \item CSCW literatura (viz issue \#9)
\end{itemize}
\end{raw}

%%%%%%%%%%%%%%%%%%%%%%%%%%%%%%%%%%%%%%%%%%%%%%%%%%%%%%%%%%%%%%%%%%%%%%%%%%%%%%%
\section{Agentic coding}
%%%%%%%%%%%%%%%%%%%%%%%%%%%%%%%%%%%%%%%%%%%%%%%%%%%%%%%%%%%%%%%%%%%%%%%%%%%%%%%

\begin{raw}
\textbf{Co může sekce obsahovat:}
\begin{itemize}
    \item Co jsou coding agents (definice, typy)
    \item Jak agenti mění vývoj (co zůstává, co se mění)
    \item Konvergence SDLC - micro-waterfall hypotéza
    \item Trade-off agenti vs týmy
\end{itemize}

\rule{\textwidth}{0.4pt}

\textbf{[K ROZPRACOVÁNÍ] Konvergence SDLC modelů - micro-waterfall hypotéza:}

Zajímavý postřeh: nevracíme se s AI agenty zpět k waterfall, jen v menší časové škále?

\begin{itemize}
    \item \textbf{Waterfall (klasický):} Requirements → Design → Implementation → Testing → Deployment [měsíce per fáze]
    \item \textbf{Agile (sprint):} Planning → Dev → Testing → Review [2-4 týdny per cyklus]
    \item \textbf{AI agenti:} Prompt → Generate → Review → Fix [minuty per cyklus]
\end{itemize}

Ve všech případech máš sekvenční kroky (specifikace → implementace → validace). Agile je nezrušil - jen zmenšil a zrychlil. AI agenti je zmenšují ještě víc.

\textbf{Klíčové rozdíly:}
\begin{itemize}
    \item Batch size: celý projekt → feature → jeden prompt
    \item Feedback loop: měsíce → týdny → minuty
    \item Kdo specifikuje: analytik (dokument) → PO + tým (stories) → developer (prompt)
    \item Kdo validuje: QA na konci → tým průběžně → developer okamžitě
\end{itemize}

\rule{\textwidth}{0.4pt}

\textbf{[K ROZPRACOVÁNÍ] Trade-off agenti vs týmy:}
\begin{itemize}
    \item Tradiční SWE = hodně o organizaci a lidech (komunikace, koordinace, předávání znalostí)
    \item Velké týmy = komunikační režie (Brooks's Law)
    \item Agenti nepotřebují meetingy, ale nerozumí nuancím - vše musí být explicitní
    \item Žádné implicitní porozumění z konverzace ("řekli jsme si na meetingu")
    \item Hypotéza: malé týmy s agenty mohou růst rychleji než velké týmy bez nich
\end{itemize}

\textbf{Zdroje:}
\begin{itemize}
    \item Surveys o LLM agentech v SE (máme v literatuře)
    \item Jin et al. - LLM Agents for SWE Survey
    \item Foundational Pillars of Agentic SE
\end{itemize}
\end{raw}

%%%%%%%%%%%%%%%%%%%%%%%%%%%%%%%%%%%%%%%%%%%%%%%%%%%%%%%%%%%%%%%%%%%%%%%%%%%%%%%
\section{Scaffolding pro agenty}
%%%%%%%%%%%%%%%%%%%%%%%%%%%%%%%%%%%%%%%%%%%%%%%%%%%%%%%%%%%%%%%%%%%%%%%%%%%%%%%

\begin{raw}
\textbf{Co může sekce obsahovat:}
\begin{itemize}
    \item Co je scaffolding (definice, proč je potřeba)
    \item Spec-driven development (BMAD, SpecKit...)
    \item Nástroje pro řízení agentů (CLAUDE.md, hooks, agents.md)
    \item Metriky a hodnocení (jak měřit úspěšnost agentů)
\end{itemize}

\rule{\textwidth}{0.4pt}

\textbf{Poznámky k zapracování:}
\begin{itemize}
    \item ITIL/CMMI relevance - ověřit jestli vůbec patří do BP
    \item Viz handoffs/03-agent-framework-brainstorm.md a 06-agent-framework-consolidated.md
\end{itemize}
\end{raw}
