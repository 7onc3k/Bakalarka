\chapter{Teoretická východiska}
\label{kap:teorie}

\section{Softwarový vývoj a jeho řízení}

\subsection{Tradiční SWE}

\begin{raw}
\textbf{Co sem patří:}
\begin{itemize}
    \item Co je SWE, SDLC, fáze vývoje
    \item Nástroje pro řízení (BPMN, dokumentace, CI/CD)
    \item Proč je řízení důležité
    \item SWEBOK jako reference
\end{itemize}
\end{raw}

\subsection{SWE s coding agents}

\begin{raw}
\textbf{Co sem patří:}
\begin{itemize}
    \item Jak agenti mění vývoj
    \item Nástroje pro řízení agentů (agents.md, skills.md, CLAUDE.md, hooks)
    \item Co je stejné, co je jiné oproti tradičnímu SWE
    \item Zdroje: surveys o LLM agentech v SE (máme v literatuře)
\end{itemize}
\end{raw}

\section{Scaffolding pro agenty}

\begin{raw}
\textbf{Co sem patří:}
\begin{itemize}
    \item Co je scaffolding (definice)
    \item Jak propojit nástroje s agenty
    \item Co agent potřebuje aby fungoval dobře
    \item Spec-driven development (BMAD, SpecKit...)
    \item Metriky pro hodnocení agentů
\end{itemize}

\textbf{Poznámky k zapracování:}
\begin{itemize}
    \item ITIL/CMMI relevance - ověřit jestli vůbec patří do BP
    \item Viz handoffs/03-agent-framework-brainstorm.md a 06-agent-framework-consolidated.md
\end{itemize}
\end{raw}
