\chapter{Teoretická východiska}
\label{kap:teorie}

% === ÚVOD KAPITOLY (mezi 2 a 2.1) - ROADMAP ===
\begin{draft}
Tato kapitola vysvětluje teoretická východiska práce. Nejprve je popsáno softwarové inženýrství jako disciplína, následně životní cyklus a metodiky vývoje software. Poté je zkoumáno, jak se oblast mění díky AI coding agentům, a nakonec jsou představeny prvky scaffoldingu (podpůrných struktur), které agenti využívají.
\end{draft}

%%%%%%%%%%%%%%%%%%%%%%%%%%%%%%%%%%%%%%%%%%%%%%%%%%%%%%%%%%%%%%%%%%%%%%%%%%%%%%%
\section{Softwarové inženýrství}
%%%%%%%%%%%%%%%%%%%%%%%%%%%%%%%%%%%%%%%%%%%%%%%%%%%%%%%%%%%%%%%%%%%%%%%%%%%%%%%

\begin{raw}
\textbf{Úvod sekce 2.1 (1-2 věty):}
Tato sekce definuje softwarové inženýrství, vysvětluje proč je software inherentně složitý a jak tato složitost vedla ke vzniku oboru.
\end{raw}

% === PODSEKCE 2.1.1 ===
\subsection{Definice a vymezení oboru}

\begin{raw}
\textbf{Co sem patří:}
\begin{itemize}
    \item IEEE/SWEBOK definice softwarového inženýrství
    \item Vymezení vůči příbuzným oborům (computer science, programování)
    \item Proč "inženýrství" - systematický, disciplinovaný přístup
    \item SWE je víc než psaní kódu - zahrnuje procesy, kvalitu, údržbu
\end{itemize}

\textbf{Zdroje:}
\begin{itemize}
    \item SWEBOK v4 - definice, knowledge areas
    \item IEEE Standard Glossary of Software Engineering Terminology
\end{itemize}
\end{raw}

% === PODSEKCE 2.1.2 ===
\subsection{Komplexita software}

\begin{raw}
\textbf{Co sem patří:}
\begin{itemize}
    \item \textbf{Proč je software složitý} (Brooks - No Silver Bullet):
    \begin{itemize}
        \item Essential complexity - inherentní složitost problému (business logika, requirements)
        \item Accidental complexity - složitost kterou si přidáváme (nástroje, technologie)
        \item Software je neviditelný, snadno měnitelný, nemá fyzická omezení
    \end{itemize}
\end{itemize}

\textbf{Zdroje:}
\begin{itemize}
    \item Brooks - No Silver Bullet (1986) - essential vs accidental complexity
    \item McConnell - Code Complete (cituje Brookse)
\end{itemize}
\end{raw}

% === PODSEKCE 2.1.3 ===
\subsection{Historický kontext}

\begin{raw}
\textbf{Co sem patří:}
\begin{itemize}
    \item NATO 1968 konference - "software crisis"
    \item Projekty selhávaly, překračovaly rozpočty, nedodávaly včas
    \item Potřeba inženýrské disciplíny místo ad-hoc programování
    \item Vznik SWE jako reakce na komplexitu a krizi
\end{itemize}

\textbf{Zdroje:}
\begin{itemize}
    \item NATO 1968 - historický kontext vzniku oboru
    \item Sommerville - Software Engineering (úvod, historie)
\end{itemize}
\end{raw}

%%%%%%%%%%%%%%%%%%%%%%%%%%%%%%%%%%%%%%%%%%%%%%%%%%%%%%%%%%%%%%%%%%%%%%%%%%%%%%%
\section{Životní cyklus a metodiky}
%%%%%%%%%%%%%%%%%%%%%%%%%%%%%%%%%%%%%%%%%%%%%%%%%%%%%%%%%%%%%%%%%%%%%%%%%%%%%%%

\begin{raw}
\textbf{Úvod sekce 2.2 (1-2 věty):}
Tato sekce popisuje jak se software vyvíjí v praxi: co se dělá (fáze), jak se to organizuje (metodiky), čím se to dělá (nástroje), co vzniká (artefakty), a kdo to dělá (role a komunikace).
\end{raw}

% === PODSEKCE 2.2.1 ===
\subsection{Fáze životního cyklu}

\begin{raw}
\textbf{Co sem patří:}
\begin{itemize}
    \item Sommerville - 4 základní aktivity:
    \begin{enumerate}
        \item Software specification
        \item Software development
        \item Software validation
        \item Software evolution
    \end{enumerate}
    \item Každá fáze produkuje artefakty
    \item Cyklický charakter - software se neustále vyvíjí
\end{itemize}

\textbf{Zdroje:}
\begin{itemize}
    \item Sommerville - Software Engineering (SDLC)
    \item SWEBOK v4 - Software Engineering Process KA
\end{itemize}
\end{raw}

% === PODSEKCE 2.2.2 ===
\subsection{Modely a metodiky}

\begin{raw}
\textbf{Co sem patří:}
\begin{itemize}
    \item Waterfall - sekvenční, dokumentace dopředu
    \item Iterativní/inkrementální - opakované cykly
    \item Agile - flexibilita, rychlý feedback, spolupráce
    \item Různé modely = různé způsoby organizace stejných fází
    \item Trade-offs: prediktabilita vs flexibilita
\end{itemize}

\textbf{Zdroje:}
\begin{itemize}
    \item Sommerville - process models
    \item Boehm - Spiral Model (1988)
    \item Beck - Extreme Programming Explained
    \item Agile Manifesto
\end{itemize}
\end{raw}

% === PODSEKCE 2.2.3 ===
\subsection{Nástroje}

\begin{raw}
\textbf{Co sem patří:}
\begin{itemize}
    \item IDE - integrovaná vývojová prostředí
    \item Version control (Git) - správa verzí, spolupráce
    \item CI/CD - automatizace buildů, testů, deploymentu
    \item Issue tracking, dokumentace
    \item Nástroje jako součást procesů - umožňují škálování
\end{itemize}

\textbf{Proč je to důležité pro BP:}
Agenti využívají tyto nástroje - musí s nimi umět pracovat. Scaffolding staví na existujících nástrojích.

\textbf{Zdroje:}
\begin{itemize}
    \item SWEBOK v4 - Software Engineering Tools KA
    \item Praktické znalosti (Git, GitHub Actions, etc.)
\end{itemize}
\end{raw}

% === PODSEKCE 2.2.4 ===
\subsection{Artefakty}

\begin{raw}
\textbf{Co sem patří:}
\begin{itemize}
    \item Specifikace a požadavky (requirements)
    \item Návrhy a architektura (design docs)
    \item Zdrojový kód
    \item Testy (unit, integration, e2e)
    \item Dokumentace (technická, uživatelská)
    \item Artefakty jako vstup/výstup fází životního cyklu
\end{itemize}

\textbf{Proč je to důležité pro BP:}
Agenti produkují a konzumují tyto artefakty. Kvalita artefaktů ovlivňuje kvalitu výstupu agenta.

\textbf{Zdroje:}
\begin{itemize}
    \item Sommerville - Software Engineering
    \item SWEBOK v4
\end{itemize}
\end{raw}

% === PODSEKCE 2.2.5 ===
\subsection{Role a komunikace}

\begin{raw}
\textbf{Co sem patří:}
\begin{itemize}
    \item SWE je fundamentálně týmová disciplína
    \item Hodně znalostí je implicitních (v hlavách lidí, ne v dokumentaci)
    \item Brooks's Law - komunikační režie roste exponenciálně s velikostí týmu
    \item Procesy slouží ke koordinaci, konzistenci, předávání znalostí
    \item Nástroje a artefakty jako prostředky komunikace (CSCW perspektiva)
\end{itemize}

\textbf{Proč je to důležité pro BP:}
Když pak v 2.3 řekneme "agenti nepotřebují meetingy ale vše musí být explicitní" - čtenář pochopí co se mění. Implicitní znalosti v týmu → explicitní instrukce pro agenta.

\textbf{Zdroje:}
\begin{itemize}
    \item Brooks - Mythical Man-Month (Brooks's Law)
    \item Beck - Extreme Programming Explained (týmová spolupráce)
    \item CSCW literatura (viz issue \#9)
\end{itemize}
\end{raw}

%%%%%%%%%%%%%%%%%%%%%%%%%%%%%%%%%%%%%%%%%%%%%%%%%%%%%%%%%%%%%%%%%%%%%%%%%%%%%%%
\section{Agentic coding}
%%%%%%%%%%%%%%%%%%%%%%%%%%%%%%%%%%%%%%%%%%%%%%%%%%%%%%%%%%%%%%%%%%%%%%%%%%%%%%%

\begin{raw}
\textbf{Co může sekce obsahovat:}
\begin{itemize}
    \item Co jsou coding agents (definice, typy)
    \item Jak agenti mění vývoj (co zůstává, co se mění)
    \item Konvergence SDLC - micro-waterfall hypotéza
    \item Trade-off agenti vs týmy
\end{itemize}

\rule{\textwidth}{0.4pt}

\textbf{[K ROZPRACOVÁNÍ] Konvergence SDLC modelů - micro-waterfall hypotéza:}

Zajímavý postřeh: nevracíme se s AI agenty zpět k waterfall, jen v menší časové škále?

\begin{itemize}
    \item \textbf{Waterfall (klasický):} Requirements → Design → Implementation → Testing → Deployment [měsíce per fáze]
    \item \textbf{Agile (sprint):} Planning → Dev → Testing → Review [2-4 týdny per cyklus]
    \item \textbf{AI agenti:} Prompt → Generate → Review → Fix [minuty per cyklus]
\end{itemize}

Ve všech případech máš sekvenční kroky (specifikace → implementace → validace). Agile je nezrušil - jen zmenšil a zrychlil. AI agenti je zmenšují ještě víc.

\textbf{Klíčové rozdíly:}
\begin{itemize}
    \item Batch size: celý projekt → feature → jeden prompt
    \item Feedback loop: měsíce → týdny → minuty
    \item Kdo specifikuje: analytik (dokument) → PO + tým (stories) → developer (prompt)
    \item Kdo validuje: QA na konci → tým průběžně → developer okamžitě
\end{itemize}

\rule{\textwidth}{0.4pt}

\textbf{[K ROZPRACOVÁNÍ] Trade-off agenti vs týmy:}
\begin{itemize}
    \item Tradiční SWE = hodně o organizaci a lidech (komunikace, koordinace, předávání znalostí)
    \item Velké týmy = komunikační režie (Brooks's Law)
    \item Agenti nepotřebují meetingy, ale nerozumí nuancím - vše musí být explicitní
    \item Žádné implicitní porozumění z konverzace ("řekli jsme si na meetingu")
    \item Hypotéza: malé týmy s agenty mohou růst rychleji než velké týmy bez nich
\end{itemize}

\textbf{Zdroje:}
\begin{itemize}
    \item Surveys o LLM agentech v SE (máme v literatuře)
    \item Jin et al. - LLM Agents for SWE Survey
    \item Foundational Pillars of Agentic SE
\end{itemize}
\end{raw}

%%%%%%%%%%%%%%%%%%%%%%%%%%%%%%%%%%%%%%%%%%%%%%%%%%%%%%%%%%%%%%%%%%%%%%%%%%%%%%%
\section{Scaffolding pro agenty}
%%%%%%%%%%%%%%%%%%%%%%%%%%%%%%%%%%%%%%%%%%%%%%%%%%%%%%%%%%%%%%%%%%%%%%%%%%%%%%%

\begin{raw}
\textbf{Co může sekce obsahovat:}
\begin{itemize}
    \item Co je scaffolding (definice, proč je potřeba)
    \item Spec-driven development (BMAD, SpecKit...)
    \item Nástroje pro řízení agentů (CLAUDE.md, hooks, agents.md)
    \item Metriky a hodnocení (jak měřit úspěšnost agentů)
\end{itemize}

\rule{\textwidth}{0.4pt}

\textbf{Poznámky k zapracování:}
\begin{itemize}
    \item ITIL/CMMI relevance - ověřit jestli vůbec patří do BP
    \item Viz handoffs/03-agent-framework-brainstorm.md a 06-agent-framework-consolidated.md
\end{itemize}
\end{raw}
